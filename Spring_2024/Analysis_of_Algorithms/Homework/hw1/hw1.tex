\documentclass{article}
\usepackage{amsmath,amssymb,amsthm,latexsym,paralist,enumitem}

\DeclareRobustCommand{\stirling}{\genfrac\{\}{0pt}{}}

\theoremstyle{definition}
\newtheorem{problem}{Problem}
\newtheorem*{solution}{Solution}
\newtheorem*{resources}{Resources}

\newcommand{\name}[1]{\noindent\textbf{Name: #1}}
\newcommand{\honor}{\noindent On my honor, as an Aggie, I have neither
  given nor received any unauthorized aid on any portion of the
  academic work included in this assignment. Furthermore, I have
  disclosed all resources (people, books, web sites, etc.) that have
  been used to prepare this homework. The work shown here is entirely
  my own, and is written in my own words.\\[1ex]
 \textbf{Signature:} \underline{Jawahar Sai Nathani}{\hspace*{5cm}} }

 
\newcommand{\checklist}{\noindent\textbf{Checklist:}
\begin{compactitem}[$\Box$] 
\item[$\checkmark$] Did you add your name? 
\item[$\checkmark$] Did you disclose all resources that you have used? \\
(This includes all people, books, websites, etc. that you have consulted)
\item[$\checkmark$] Did you follow the Aggie honor code? 
\item[$\checkmark$] Did you solve all problems? 
\item[$\checkmark$] Did you submit the pdf file (resulting from your latex file)
  of your homework?
\end{compactitem}
}

\newcommand{\problemset}[1]{\begin{center}\textbf{Problem Set #1}\end{center}}
\newcommand{\duedate}[1]{\begin{quote}\textbf{Due date:} Electronic
    submission the .pdf file of this homework is due on \textbf{#1} on canvas
    (as a turnitin assignment).\end{quote}}

\newcommand{\N}{\mathbf{N}}
\newcommand{\R}{\mathbf{R}}
\newcommand{\Z}{\mathbf{Z}}

\begin{document}
\problemset{1}
\centerline{CSCE 411/629 (Dr. Klappenecker) }

\duedate{Friday, 1/26, 11:59pm}
\name{ Jawahar Sai Nathani}
\begin{resources} (All people, books, articles, web pages, etc. that
  have been consulted when producing your answers to this homework)
\end{resources}
\honor

\newpage%
\noindent Get familiar with \LaTeX. All exercises in this homework are from the lecture
notes on perusall, not from our textbook. \medskip

\noindent\textbf{Reading assignment:} Carefully read the lecture notes
\verb|dm_ch11.pdf| on Perusall. Skim Appendices A and B in the textbook. 

\begin{problem}
Exercise 11.3 (in notes on perusall) \\
Ernie and Bert study the functions $f(n) = n^2 + 2n$ and $g(n) = n^2$. Ernie claims that the functions are asymptotically equal, so $f \sim g$. Bert insists that f can never be asymptotically equal to g, since they always differ quite significantly, namely $f(n) - g(n) \geq 2n$ for all $n \geq 1$. Explain why one is right and why the other is wrong.
\begin{solution}
 According to Asymptotic Equality definition, we write $f \sim g$ and say that $f$ is \textbf{asymptotically equal} to $g$ if 
\[ \lim_{n \to \infty} \frac{f(n)}{g(n)} = 1\]
We have $f(n) = n^2 + 2n$ and $g(n) = n^2$
\[ \lim_{n \to \infty} \frac{f(n)}{g(n)} = \lim_{n \to \infty} \frac{n^2 + 2n}{n^2} = 1 + \lim_{n \to \infty} \frac{2}{n} \]
As $n \to \infty$, $\frac{2}{n} \to 0$, Therefore
\[ \lim_{n \to \infty} \frac{f(n)}{g(n)} = 1 + 0 = 1 \]
Which satisfies the above definition. Hence $f$ and $g$ are \textbf{asymptotically equal}. Ernie's claim is correct.
\end{solution}
\end{problem}

\begin{problem}
Exercise 11.9 (in notes on perusall) \\
Find all accumulation points of the following functions, and determine their upper and lower accumulation points.
\begin{enumerate}[label=(\alph*)]
\item $f(n) = (-1)^n$,
\item $f(n) = 4 + (-1)^n/(n+10)$,
\item $f(n) = ((-1)^n+(-1)^{\lfloor{n/2}\rfloor})(1 + 1/n)$.
\end{enumerate}
\begin{solution}
\begin{enumerate}[label=(\Alph*)]
\item Given $f(n) = (-1)^n$ , which can be written as 
\begin{equation}
  f(n)=\begin{cases}
    1 & \text{if $n$ is even}.\\
    -1 & \text{if $n$ is odd}.
  \end{cases}
\end{equation}
\begin{itemize}
  \item For any $\epsilon > 0$, for all even positive integers n $f(n) = 1$. As the function has an infinite number of function values satisfying $$|f(n) - 1| = |1-1| = 0 < \epsilon$$ Hence, a = 1 is an accumulation point.
  \item For any $\epsilon > 0$, for all odd positive integers n $f(n) = -1$. As the function has an infinite number of function values satisfying $$|f(n) - (-1)| = |-1-(-1)| = 0 < \epsilon$$ Hence, a = -1 is an accumulation point.
  \item For any $\epsilon > 0$, for all even positive integers n we have infinitely many function values satisfying $f(n) > 1 - \epsilon$. We have $f(n) > 1 + \epsilon$ for zero function values. Hence, 1 is an upper accumulation point.
  \item For any $\epsilon > 0$, for all odd positive integers n we have infinitely many function values satisfying $f(n) < -1 + \epsilon$. We have $f(n) < -1 - \epsilon$ for zero function values. Hence, -1 is a lower accumulation point.
\end{itemize}
Therefore, \textbf{1, -1 are accumulation points} of function $f(n)$ with \textbf{1 as upper accumulation point} and \textbf{-1 as lower accumulation point.}

\item Given $f(n) = 4 + (-1)^nn/(n+10)$, which can be written as
\begin{equation}
  f(n)=\begin{cases}
    4 + n/(n+10) & \text{if $n$ is even}.\\
    4 - n/(n+10) & \text{if $n$ is odd}.
  \end{cases}
\end{equation}
Similar to the above solution (a), as $\lim_{n\to\infty}n/(n+10) = 1$ this function has 4+1 = 5 and 4-1 = 3 as accumulation points.
\begin{itemize}
    \item For any $\epsilon > 0$, for all even positive integers $n > 10(1/\epsilon - 1)$ we have infinitely many function values satisfying $f(n) > 5 - \epsilon$. We have $f(n) > 5 + \epsilon$ for no function value. Hence, 5 is an upper accumulation point.
    \item For any $\epsilon > 0$, for all odd positive integers $n > 10(1/\epsilon - 1)$ we have infinitely many function values satisfying $f(n) < 3 + \epsilon$. We have $f(n) < 3 - \epsilon$ for no function value. Hence, 3 is a lower accumulation point.
\end{itemize}
Therefore, \textbf{5, 3 are accumulation points} of function $f(n)$ with \textbf{5 as upper accumulation point} and \textbf{3 as lower accumulation point.}

\item Given $f(n) = ((-1)^n+(-1)^{\lfloor{n/2}\rfloor})(1 + 1/n)$, which can be written as
\begin{equation}
  f(n)=\begin{cases}
    -2 - 2/n & \text{if $n = 4K - 1$ and $K \in I$}.\\
    2 + 2/n & \text{if $n = 4K$ and $K \in I$}.\\
    0 & \text{if $n = 4K + 1$ and $K \in I$}.\\
    0 & \text{if $n = 4K + 2$ and $K \in I$}.\\
  \end{cases}
\end{equation}
Similar to the above solution (a), as $\lim_{n\to\infty}1/n = 0$ this function has -2-0 = -2, 2+0 = 2 and 0 as accumulation points. \\
Therefore, \textbf{2, 0 and -2 are accumulation points} of function $f(n)$ with \textbf{2 as upper accumulation point} and \textbf{-2 as lower accumulation point.}
\end{enumerate}
\end{solution}
\end{problem}

\begin{problem}
Exercise 11.17 (in notes on perusall)\\
Let b and d be positive real numbers that are not equal to 1.
\begin{enumerate}[label=(\alph*)]
\item Show that $\Theta(\log_bn) = \Theta(\log_dn)$, so one can write $\Theta(\log{n})$ using a baseless logarithm without causing confusion.
\item Prove or disprove: Does $\Theta({n^{\log_bn}}) = \Theta({n^{\log_dn}})$ hold in general?
\end{enumerate}
\begin{solution}
\begin{enumerate}[label=(\Alph*)]
\item We need to prove $\Theta(\log_bn) = \Theta(\log_dn)$. Let $f(n) = \log_bn$ and $g(n) = \log_dn$
$$\lim_{n\to\infty}\frac{\lvert f(n) \rvert}{\lvert g(n) \rvert} = \lim_{n\to\infty}\frac{\log_b n}{\log_d n} = \lim_{n\to\infty}\frac{(\frac{\log n}{\log b})}{(\frac{\log n}{\log d})} = \lim_{n\to\infty}\frac{\log d}{\log b} = \frac{\log d}{\log b}$$
Therefore, $f(n) = \Theta(g(n))$ and $g(n) = \Theta(f(n)) \Longrightarrow \Theta(\log_bn) = \Theta(\log_dn)$. Hence Proved. 

\item We need to prove $\Theta({n^{\log_bn}}) = \Theta({n^{\log_dn}})$. If we take limit as n $\to\infty$ $\lvert{f(n)}\rvert/\lvert{g(n)}\rvert$
$$\lim_{n\to\infty}\frac{\lvert f(n) \rvert}{\lvert g(n) \rvert} = \lim_{n\to\infty}\frac{n^{\log_bn}}{n^{\log_dn}}$$
The above limit exists only if b = d and the limit is undefined for all other values of b and d. Therefore, $\Theta(\log_bn) = \Theta(\log_dn)$ doesn't hold in general.
\end{enumerate}
\end{solution}
\end{problem}

\begin{problem}
Exercise 11.19 (in notes on perusall) \\
Show that for all positive integers k, we have
$$1^k + 2^k + \cdots + n^k = \Theta(n^{k+1})$$
\begin{solution}
Need to prove $1^k + 2^k + \cdots + n^k = \Theta(n^{k+1})$ \\
Let $f(n) = 1^k + 2^k + \cdots + n^k$ and $g(n) = n^{k+1}$
$$\lim_{n\to\infty}\frac{\lvert f(n) \rvert}{\lvert g(n) \rvert} = \lim_{n\to\infty}\frac{1^k + 2^k + \cdots + n^k}{n^{k+1}} = \lim_{n\to\infty}\frac{1}{n}((1/n)^k + (2/n)^k + \cdots + (n/n)^k)$$
$$= \lim_{n\to\infty}\frac{1}{n}\sum_{k=1}^{n} (\frac{r}{n})^k = \int_{0}^{1} x^k \,dx = \frac{1}{k+1}$$
As K is a positive integer and $\lim_{n\to\infty}{\lvert f(n) \rvert}/{\lvert g(n) \rvert}$ exists. \\ Therefore, $1^k + 2^k + \cdots + n^k = \Theta(n^{k+1})$
\end{solution}
\end{problem}

\begin{problem}
Exercise 11.35 (in notes on perusall) \\
Show that $f \in O(g)$ and $f \notin o(g)$ does not imply that $f \in \Theta(g)$.
\begin{solution}
Given $f \in O(g)$ and $f \notin o(g)$. Let us take an example that satisfies these 2 conditions.
$$g(n) = 1 \hspace{3mm} and \hspace{3mm} f(n) = 1 + (-1)^n$$
\begin{equation}
  f(n)=\begin{cases}
    2 & \text{if n is even}.\\
    0 & \text{if n is odd}\\
  \end{cases}
\end{equation}
\begin{itemize}
\item There exists some positive constant C i.e 3 such that \\
$\lvert f(n) \rvert \leq C \lvert g(n) \rvert$, $\forall n > n_0$. This satisfies $f \in O(g)$
\item There exists no $n_0$, where for every $\epsilon > 0$ the inequality \\
$\lvert f(n) \rvert \leq \epsilon \lvert g(n) \rvert$, $\forall n > n_0$ holds. i.e For $\epsilon = 1$, $n = 2$ inequality doesn't hold as $2 \nleqslant 1$. This satisfies $f \notin o(g)$.
\item There exists no positive constant c such that $c\lvert g(n) \rvert \leq \lvert f(n) \rvert$, $\forall n > n_0$. Because $c \nleqslant 0$, which doesn't satisfy the basic rule of $f = \Theta{(g)}$.
\end{itemize}
Therefore, $f \in O(g)$ and $f \notin o(g)$ does not imply $f = \Theta{(g)}$. Hence Proved.
\end{solution}
\end{problem}

\noindent\rule{12.2cm}{0.4pt} \\

Homeworks must be typeset in \LaTeX{}. The solution must be written in
your own words. 









\goodbreak
\checklist
\end{document}
