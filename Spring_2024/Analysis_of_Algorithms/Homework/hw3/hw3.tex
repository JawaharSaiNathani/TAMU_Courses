\documentclass{article}
\usepackage{amsmath,amssymb,amsthm,latexsym,paralist}

\theoremstyle{definition}
\newtheorem{problem}{Problem}
\newtheorem*{solution}{Solution}
\newtheorem*{resources}{Resources}

\newcommand{\name}[1]{\noindent\textbf{Name: #1}}
\newcommand{\honor}{\noindent On my honor, as an Aggie, I have neither
  given nor received any unauthorized aid on any portion of the
  academic work included in this assignment. Furthermore, I have
  disclosed all resources (people, books, web sites, etc.) that have
  been used to prepare this homework. \\[1ex]
 \textbf{Signature:} \underline{Jawahar Sai Nathani}{\hspace*{5cm}} }

 \newcommand{\checklist}{\noindent\textbf{Checklist:}
\begin{compactitem}[$\Box$] 
\item [$\checkmark$] Did you add your name? 
\item [$\checkmark$] Did you disclose all resources that you have used? \\
(This includes all people, books, websites, etc. that you have consulted)
\item [$\checkmark$] Did you sign that you followed the Aggie honor code? 
\item [$\checkmark$] Did you solve all problems? 
\item [$\checkmark$] Did you submit the pdf file of your homework? Check!
\end{compactitem}
}



\newcommand{\problemset}[1]{\begin{center}\textbf{Problem Set
      #1}\end{center}}
\newcommand{\duedate}[1]{\begin{quote}\textbf{Due date:} Electronic
    submission of the pdf file of this homework is due on
    \textbf{#1} on ecampus. \end{quote} }

\newcommand{\N}{\mathbf{N}}
\newcommand{\R}{\mathbf{R}}
\newcommand{\Z}{\mathbf{Z}}


\begin{document}
\problemset{3}
\duedate{2/9/2023 before 11:59pm}
\name{Jawahar Sai Nathani}
\begin{resources} (All people, books, articles, web pages, etc. that
  have been consulted when producing your answers to this homework)
\end{resources}
\honor

\newpage
\noindent Make sure that you describe all solutions in your own words. \\[1ex]
Read chapters 2 and 4 in our textbook before attempting to solve these
problems. 

\medskip

\begin{problem}[20 points]
  Use mathematical induction to show that when $n$ is an exact power of 2, the solution of the recurrence
$$ T(n) = \begin{cases}
2 & \text{ if $n=2$,} \\
2T(n/2) + n & \text{if $n=2^k$ for $k>1$,}
\end{cases}
$$
is $T(n)=n\log_2 n$. 
\end{problem}
\begin{solution}
First let's check for base cases when n = 2 and 4 \\
\underline{n = 2} \\
$T(n) = n\log_2n \\ \Rightarrow T(2) = 2\log_22 \\ \Rightarrow T(2) = 2$ \\
\underline{n = 4} \\
$T(4) = 4\log_24 = 4*2 = 8$\\
According to the recurrence $T(4) = 2T(4/2) + 4 \\ T(4) = 2*T(2) + 4 = 2*2 + 4 = 8$ \\
Relationship holds for both n = 2 and 4.\\
% \bigbreak
\textbf{\underline{Inductive Case}} \\
let's assume $T(n) = n\log_2n$, when $n = 2^k.$ For $n = 2^{k+1}$ \\
\begin{flalign*}
T(2^{k+1})  &= 2T(2^{k+1}/2)+2^{k+1}    &\\
            &= 2T(2^k)+2^{k+1}          &\\
            &\text{substitute}\hspace{2mm} T(2^k) = 2^k\log_22^k \hspace{2mm}\text{according to our assumption}&\\
            &= 2\left(2^k\log_2\left(2^k\right)\right)+2^{k+1} &\\
            &= 2^{k+1}\log_2(2^k)+2^{k+1}   &\\
            &= 2^{k+1}\left(\log_2(2^k)+1\right)    &\\
            &= 2^{k+1}\left(\log_2(2^k)+\log_22\right)    &\\
            &= 2^{k+1}\log_2(2^{k+1}) &\\
            &= n\log_2(n)
\end{flalign*}
Therefore, $T(n) = n\log_2n$ for the above recurrence holds.
\end{solution}
\newpage

\begin{problem}[20 points]
  We can express insertion sort as a recursive procedure as follows. 
In order to sort $A[1..n]$, we recursively sort $A[1..n-1]$ and then
insert $A[n]$ into the sorted array $A[1..n-1]$. Write a recurrence for
the running time of this recursive version of insertion sort. 
\end{problem}
\begin{solution}
In insertion sort, we first sort the starting $n-1$ elements and then insert A[n] into the sorted array.\\
To sort the array of $n-1$ elements it takes $T(n-1)$ running time and to insert A[n] into the sorted array, it takes $\Theta(n)$ runtime, since we need to compare A[n] with all the elements in the array to find the correct index. \\
Hence, insertion sort runtime for an array with n elements is 
$$T(n) = T(n-1) + \Theta(n)$$
Whereas $n = 1$ is a special case. For $n = 1$ since there are no other elements to sort, we just need to insert the element into the array. Runtime for $n = 1$ is $\Theta(1)$. Therefore
$$ T(n) = \begin{cases}
\Theta(1) & \text{ if $n=1$,} \\
T(n-1) + \Theta(n) & \text{if $n>1$}
\end{cases}
$$
\end{solution}

\begin{problem}[20 points]
  V. Pan has discovered a way of multiplying $68 \times 68$ matrices
  using $132,464$ multiplications, a way of multiplying $70\times 70$
  matrices using $143,640$ multiplications, and a way of multiplying
  $72\times 72$ matrices using $155,424$ multiplications. Which method
  yields the best asymptotic running time when used in a
  divide-and-conquer matrix-multiplication algorithm? How does it
  compare to Strassen’s algorithm?
\end{problem}
\begin{solution}
The recurrence rule for time complexity of an algorithm with 'b' sub-problems and 'a' multiplications is
$$T(n) = aT(n/b)$$
Applying master theorem, T(n) = $\Theta(n^{\log_ba})$. Time complexity for the first method is \\
$\Theta(n^{\log_{68}132464})$ and $\log_{68}132464 \approx 2.7951284874$ \\
Time complexity for 2nd method is \\
$\Theta(n^{\log_{70}143640})$ and $\log_{70}143640 \approx 2.7951226897$ \\
Time complexity for 3nd method is \\
$\Theta(n^{\log_{72}155424})$ and $\log_{72}155424 \approx 2.7951473911$ \\
and Time complexity for Strassen's algorithm with 7 multiplications and 2 sub-processes is\\
$\Theta(n^{\log_{2}7})$ and $\log_{2}7 \approx 2.81$ \\
Therefore, method 2 yields the best asymptotic running time and is better than Strassen's algorithm.
\end{solution}
\newpage
\begin{problem}[20 points]
  Show how to multiply the complex numbers $a+bi$ and $c+di$ using
  only three multiplications of real numbers. The algorithm should
  take $a$, $b$, $c$, and $d$ as input and produce the real component
  $ac-bd$ and the imaginary component $ad+bc$ separately. [Hint: First
  study Karatsuba's integer multiplication algorithm.]
\end{problem}
\begin{solution}
Let's consider \\
$P_1 = (a+b)(c-d) = ac - ad + bc - bd$ \\
$P_2 = ad$\\
$P_3 = bc$\\
\begin{flalign*}
P_1 + P_2 - P_3  &= ac - ad + bc - bd + ad - bc    &\\
            &= ac - bd          &\\
            &= \text{real component} &\\
P_2 + P_3   &= ad + bc   &\\
            &= \text{imaginary component}  
\end{flalign*}
$\Rightarrow (a + bi)(c + di) = (P_1 + P_2 - P_3) + i(P_2 + P_3) $ \\
$P_1, P_2$ and $P_3$ all have one multiplication in each. Therefore, 2 complex numbers can be multiplied only using 3 multiplications following the above method. 
\end{solution}

\begin{problem}[20 points]
Use the master method to show that the solution to the binary-search 
recurrence 
$$ T(n) = T(n/2)+\Theta(1)$$
is $T(n) = \Theta(\lg n)$. Clearly indicate which case of the Master
theorem is used.  
\end{problem}
\begin{solution}
$T(n) = aT(n/b) + f(n)$ when $a = 1, b = 2, f(n) = \Theta(1)$. \\
$\Theta(n^{\log_ba}) = \Theta(n^{\log_21}) = \Theta(n^0) = \Theta(1)$\\
$f(n) = \Theta(n^{\log_ba})$. \\
According to the second asymptotic bound of Master's theorem if $f(n) = \Theta(n^{\log_ba})$ then $T(n) = \Theta(n^{log_ba}\lg{n})$\\
$\Rightarrow T(n) = \Theta(n^{\log_21}\lg{n}) = \Theta(1*\lg{n}) = \Theta(\lg{n})$. \\
Therefore, $T(n) = \Theta(\lg n)$.
\end{solution}

Work out your own solutions, unless you want to risk an honors
violation!
\medskip



\goodbreak
\checklist
\end{document}
